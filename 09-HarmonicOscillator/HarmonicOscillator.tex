% This work is licensed under the Creative Commons Attribution-NonCommercial 4.0 International License.
% To view a copy of this license, visit http://creativecommons.org/licenses/by-nc/4.0/
% or send a letter to Creative Commons, PO Box 1866, Mountain View, CA 94042, USA.

% !TEX TS-program = xelatex

\documentclass[../Main/chem331-notes.tex]{subfiles}
\begin{document}

\setcounter{section}{8}

\section{The harmonic oscillator}
\subsection{Classical oscillations}
In classical physics the most simple example of period oscillatory motion is given by the harmonic oscillator.
A harmonic oscillator has a potential that is quadratic with respect to the position
\begin{equation}
V(x) = \frac{1}{2} k x^2,
\end{equation}
and a force that is linear
\begin{equation}
F = -\frac{dV(x)}{dx} = -k x.
\end{equation}
This last equation shows that the harmonic oscillator is like a spring. The more you pull it, the larger is the force that tries to restore the position of the particle to $x = 0$.

Newton's equation for a particle of mass $m$ that experiences a harmonic potential is
\begin{equation}
F = ma \Rightarrow   - k x = m \frac{d^2 x}{dt^2},
\end{equation}
and its solutions have the form
\begin{equation}
x(t) = A \cos(\omega t + B),
\end{equation}
where 
\begin{equation}
\omega = \sqrt{\frac{k}{m}},
\end{equation}
and $A$ is the amplitude of the oscillation (the maximum value of $x(t)$ possible) and $B$ is a phase factor.
The classical solution corresponds to periodic oscillation of the particle with frequency $\omega / 2\pi$.\mnote{Because it take a time $T = 2\pi/\omega$ to complete a full oscillation.}
Both $A$ and $B$ depend on the initial conditions. For example, for a harmonic oscillator that at $t = 0$ has zero velocity and is at position $x_0$, then we have that $A = x_0$ and $B = 0$, and the solution is
\begin{equation}
x(t) = x_0 \cos(\omega t).
\end{equation}


\subsection{Harmonic oscillator as a model for vibrations in diatomics}
Our main motivation for studying the harmonic oscillator is having a model for vibrations of atoms in molecules.
To see the connection between these two problems we will consider a diatomic molecule, A--B, with general molecular potential $V(r)$, where $r$ is the distance between atoms A and B.
If A and B form a bond, this potential will have a minimum at a certain bond length $r_e$, the \textbf{equilibrium bond distance}.
Since $V(r)$ has a minimum at $r_e$, its first derivative evaluated at $r_e$ is zero
\begin{equation}
\label{eq:harmonic:stationary}
\left.\frac{dV(r)}{dr}\right|_{r = r_e}  = 0.
\end{equation}
To show the connection between the potential $V(r)$ and the harmonic potential we Taylor expand the former around the point $r = r_e$
\begin{equation}
V(r) = V(r_e) + \left.\frac{dV(r)}{dr}\right|_{r = r_e} (r - r_e) 
+ \frac{1}{2} \left.\frac{d^2V(r)}{dr^2}\right|_{r = r_e} (r - r_e)^2
+ \text{higher order terms},
\end{equation}
where the higher order terms include the third and higher powers of $(r - r_e)$.
If we truncate the Taylor expansion to second order terms and used Eq.~\eqref{eq:harmonic:stationary} we can rewrite it as
\begin{equation}
V(r) \approx V(r_e) 
+ \frac{1}{2} k (r - r_e)^2,
\end{equation}
where $k$ is the second derivative of the potential at the equilibrium bond distance
\begin{iequation}
\label{eq:harmonic:fc}
k = \left.\frac{d^2V(r)}{dr^2}\right|_{r = r_e}.
\end{iequation}

Next, we can introduce a new variable,
\begin{equation}
x = r - r_e,
\end{equation}
which measures the displacement from the equilibrium bond distance, and express the potential as a function of $x$. This step leads to
\begin{equation}
V(x + r_e) = V(r_e) + k x^2.
\end{equation}
It is convenient at this point to rewrite this function is a slightly simpler way. Let us define a function $\tilde{V}(x) = V(x + r_e)$. For this new function $\tilde{V}(0) = V(r_e)$, and so we can rewrite the potential expanded to second order as
\begin{equation}
\tilde{V}(x) = \tilde{V}(0) + \frac{1}{2} k x^2.
\end{equation}
What we have done here is just express the potential as a function of the variable $x$ instead of $r$. This is just a change of variable, as you have already seen as a technique to simplify integrals.
At this point we can see that the potential $V(r)$ approximated with a Taylor expansion around the equilibrium bond distance simply gives us a harmonic potential that contains a constant term $\tilde{V}(0)$ and with force constant $k$ given by Eq.~\eqref{eq:harmonic:fc}.

\subsection{Hamiltonian for the harmonic oscillator and boundary conditions}
Now we want to study the harmonic oscillator in the setting of quantum mechanics.
This simple problem will be very useful to model the oscillations of atoms in molecules.
Let us first write the Hamiltonian
\begin{equation}
\hat{H} = \hat{T} + \hat{V} = -\frac{\hbar^2}{2m} \frac{d^2}{d x^2} + \frac{1}{2} k x^2.
\end{equation}
Note that for any finite value of $x$ the potential is finite. This situation is different than that we encountered for the case of a particle in a box.
Because the potential is finite, the particle could have a finite probability of being found anywhere on the $x$ axis.
However, the wave function must go to zero when $x \rightarrow \pm \infty$ because otherwise the wave function could not be normalized.
The appropriate boundary conditions for the harmonic oscillator are then
\begin{iequation}
\lim_{x \rightarrow \pm \infty} \psi(x) = 0.
\end{iequation}

\subsection{Eigenvalues and eigenfunctions of the harmonic oscillator}
While in the case of the particle in a box we were able to directly solve the Schr\"{o}dinger equation, for the harmonic oscillator we will directly look at the solutions skipping entirely their derivation.
The eigenfunctions of the harmonic oscillator are usually written in terms of a special class of polynomials named after the mathematician Charles Hermite\mnote{Although they were invented by Pafnuty Chebyshev. Don't worry though, there are special polynomial named after Chebyshev as well.}
Hermit polynomials are generally written as $H_n(x)$, where $n = 0, 1, \ldots$ is an integer and $x$ is a real number.
They are given by the formula
\begin{equation}
H_n(x) = (-1)^n e^{x^2} \frac{d^n}{dx^n} e^{-x^2},
\end{equation}
which for the first five polynomials gives
\begin{align}
H_0(x) & = 1, \\
H_1(x) & = 2x, \\
H_2(x) & = 4x^2 -2, \\
H_3(x) & = 8x^{3} -12x, \\
H_4(x) & = 16x^{4} -48x^2 + 12.
\end{align}
It is also possible to express them using the \textbf{recursive relationship}
\begin{equation}
H_{n+1}(x) = 2x H_n(x) - 2n H_{n-1}(x),
\end{equation}
which is a recipe to compute the polynomial of order $n+1$ once you know the polynomials of order $n$ and $n-1$. Since $H_0(x) = 1$ and $H_1(x) = 2x$, we can use this equation to reconstruct the entire series of Hermit polynomials. For example,
\begin{equation}
H_{2}(x) = 2x H_1(x) - 2 H_{0}(x) = 2x (2x) - 2 = 4x^2 - 2.
\end{equation}

The eigenfunctions of the harmonic oscillator are given by
\begin{iequation}
\psi_v(x) = \frac{1}{\left(2^v v!\right)^\frac{1}{2}} \left(\frac{\alpha}{\pi}\right)^{\frac{1}{4}} H_v(\sqrt{\alpha} x) \, e^{-\alpha x^2} \quad v = 0, 1, 2, \ldots,
\end{iequation}
where $v$ is an integer quantum number and $\alpha$ is defined as
\begin{iequation}
\alpha = \left(\frac{km}{\hbar^2} \right)^\frac{1}{2} = \frac{m \omega}{\hbar}.
\end{iequation}

The eigenvalues are given by the expression
\begin{iequation}
E_v = \hbar \omega \left(v + \frac{1}{2}\right)
\end{iequation}

\subsection{Classical turning point and tunneling}
A classical harmonic oscillator with total energy $E$ will stretch until the kinetic energy is fully converted to potential energy, in which case
\begin{equation}
E = V = \frac{1}{2} k x_\mathrm{t}^2,
\end{equation}
where $x_\mathrm{t}$ is the coordinate of the \textbf{turning points}.
The turning points may be obtained by inverting the equation above
\begin{align}
x_\mathrm{t}^{+} = + \sqrt{\frac{2 E}{k}}, \\
x_\mathrm{t}^{-} = - \sqrt{\frac{2 E}{k}}.
\end{align}
A classical particle will not move beyond the turning points, that is, the probability of finding the particle above or below the classical turning points is zero.
However, for the quantum harmonic oscillator this is not true and the probabilities are finite
\begin{align}
P [(x \leq x_\mathrm{t}^{-})  \text{ or }  (x \geq x_\mathrm{t}^{+})] = \int_{-\infty}^{x_\mathrm{t}^{-}} |\psi_v(x)|^2 \, dx + \int_{x_\mathrm{t}^{+}}^{\infty} |\psi_v(x)|^2 \, dx > 0.
\end{align}



\end{document}

