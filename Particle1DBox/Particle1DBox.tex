% This work is licensed under the Creative Commons Attribution-NonCommercial 4.0 International License.
% To view a copy of this license, visit http://creativecommons.org/licenses/by-nc/4.0/
% or send a letter to Creative Commons, PO Box 1866, Mountain View, CA 94042, USA.

% !TEX TS-program = xelatex

\documentclass[../Main/chem331-notes.tex]{subfiles}
\begin{document}

\setcounter{section}{5}

\section{The particle in a one-dimensional box}
\subsection{Potential and boundary conditions}
Consider a particle of mass $m$ in a one-dimensional box of length $L$.
A potential that models the walls of a box must not allow the particle to get out of the range $0 \leq x \leq L$.
One way to achieve this is with a potential that is infinite outside the box
\begin{equation}
V(x) =
\begin{cases}
0 & \text{if}\quad 0 \leq x \leq L \\
\infty & \text{if}\quad  x < 0 \text{ or } x > L.
\end{cases}
\end{equation}
A wave function that describe the particle in the box must satisfy $\psi(x) = 0$ if $x$ is outside the box.
Because the wave function must be continuous, the value of $\psi(x)$ at the walls must also be zero.
It follows then that
\begin{iequation}
\psi(0) = \psi(L) = 0.
\end{iequation}
This is an example of \textbf{boundary conditions}.\mnote{Boundary conditions are a set of constraints considered when solving a differential equation.}

\subsection{Schr\"{o}dinger equation}
We can now write the Schr\"{o}dinger equation for the particle in a one-dimensional box.
We only consider points inside the box. In this case the Hamiltonian is given by
\begin{equation}
\hat{H} = -\frac{\hbar^2}{2m} \frac{d^2}{dx^2}.
\end{equation}
Note that the potential inside the box is zero.

The Schr\"{o}dinger equation is given by
\begin{equation}
-\frac{\hbar^2}{2m} \frac{d^2}{dx^2} \psi(x) = E \psi(x).
\end{equation}
This is a homogeneous second-order differential equation and its solution is known
\begin{equation}
\psi(x) = A \cos(kx) + B \sin(kx),
\end{equation}
and the constants $A$, $B$, and $k$ must be determined by imposing the boundary conditions.
We can very that this is a solution to the Schr\"{o}dinger equation  by plugging it in
\begin{equation}
-\frac{\hbar^2}{2m} \frac{d^2}{dx^2} \left[ A \cos(kx) + B \sin(kx) \right]
= \underbrace{\frac{\hbar^2 k^2}{2m}}_{E} \left[ A \cos(kx) + B \sin(kx) \right].
\end{equation}
We indeed see that once we apply the Hamiltonian we get back the same function times a number, which is the eigenvalue
\begin{equation}
E = \frac{\hbar^2 k^2}{2m}
\end{equation}
Imposing $\psi(x = 0) = 0$ we get
\begin{equation}
\psi(0) = A \cos(0) + B \sin(0) = A = 0,
\end{equation}
which implies that $A$ must be zero.
Imposing $\psi(x = L) = 0$ we get
\begin{equation}
\psi(L) = B \sin(k L) = 0.
\end{equation}
Recall that $\sin(x) = 0$ if $x = n \pi$ where $n$ is an integer $n= 0, \pm 1, \pm 2, \ldots$.
It follows that $k L = n \pi$, in other words
\begin{iequation}
k=\frac{n \pi}{L}
\end{iequation}
Notice that this condition imposes that $k$ is quantized.
Quantization is a consequence of the fact that we imposed boundary conditions on the Schr\"{o}dinger equation.
Now if we plug this solution in the wave function we get
\begin{equation}
\psi_{n}(x) = B \sin\left(\frac{n \pi x}{L}\right).
\end{equation}
Note that the solution with $-n$ is equivalent because it only differs in the sign\mnote{Recall our discussion of equivalent wave functions and phase factors?}
\begin{equation}
\psi_{-n}(x) = B \sin\left(\frac{-n \pi x}{L}\right) = - B \sin\left(\frac{n \pi x}{L}\right) = -\psi_{n}(x).
\end{equation}
Of these two equivalent solutions we can discard those with negative integers $n$.
Also, notice that when $n$ we get a very sad solution
\begin{equation}
\psi_{0}(x) = B \sin\left(\frac{0 \pi x}{L}\right)  = 0.
\end{equation}
As we discussed before, this is not a good solution because there is no way we can normalize it.
These considerations leaves us with the following value of the quantum number $n$
\begin{iequation}
n = 1,2,3,\ldots
\end{iequation}

\begin{exercise}
Draw the energy spectrum of the particle in the box model.
\end{exercise}

\subsection{Energy}
We can now derive an expression for the energy
\begin{equation}
E = \frac{\hbar^2 k^2}{2m} = \frac{\hbar^2 n^2 \pi^2}{2m L^2} = \frac{h^2 n^2 }{8 m L^2}, \quad n = 1,2,3\ldots
\end{equation}

\subsection{Wave function}
Next, we should check if the wave function is normalized.
To do this, we compute the integral of the wave function probability density\mnote{Here we need the integral $\int \sin(x)^2 dx = \frac{x}{2} - \frac{1}{4} \sin(2 x) + C$.}
\begin{equation}
\int_0^L dx\,|\psi_{n}(x)|^2 = |B|^2 \int_0^L dx\,\sin^2\left(\frac{n \pi x}{L}\right)
=  |B|^2 \frac{L}{2}
\end{equation}
This wave function is not normalized because the quantity $|B|^2 \frac{L}{2}$ is generally not equal to one.
However, we can pick a value of $B$ that makes the wave function normalized.
As you can easily verify, if we set $B = \sqrt{\frac{2}{L}}$ then the wave function is normalized.
We finally arrive at the normalized expression for the wave function
\begin{iequation}
\psi_{n}(x) = \sqrt{\frac{2}{L}} \sin\left(\frac{n \pi x}{L}\right).
\end{iequation}
\begin{exercise}
Draw the first four wave functions and corresponding probability distributions for the particle in a box model.
Where is the probability of finding the particle highest and the lowest?
\end{exercise}
\end{document}

