% This work is licensed under the Creative Commons Attribution-NonCommercial 4.0 International License.
% To view a copy of this license, visit http://creativecommons.org/licenses/by-nc/4.0/
% or send a letter to Creative Commons, PO Box 1866, Mountain View, CA 94042, USA.

% !TEX TS-program = xelatex
%!TEX root =  Waves_standalone.tex

\section{Properties of waves}
\subsection{Waves}
Consider a periodic wave traveling with velocity $v$ and wavelength $\lambda$.\mnote{In SI units the wavelength ($\lambda$) has units of meters (\si{\meter}) and velocity ($v$) has units of \si{\meter\per\second}.
}
 The following quantities are defined in terms of $v$ and $\lambda$:
\begin{itemize}
\item Time between peaks: $T = \lambda / v, \quad [T]$ = \si{\second}.
\item Frequency: $\nu = \frac{1}{T}=\frac{v}{\lambda},  \quad [\nu]$ = \si{\per\second}.
\item Angular frequency: $\omega = 2\pi \nu,  \quad [\omega]$ = \si{\radian\per\second}.
\end{itemize}

In empty space light travels at the speed of light (a constant), $c = 299792458 \, \si{\meter\per\second}$.
For light then wavelength and frequency are related by:
\begin{equation}
%\tcboxmath{
\lambda \nu = c.
%}
\end{equation}

\subsection{Blackbody radiation}
According to classical physics, an idealized blackbody at temperature $T$ emits radiation with a given energy density
$\rho(\nu,T)$, where $\nu$ is the frequency of the light emitted.
The quantity $\rho(\nu,T) d\nu$ is the total energy emitted per unit volume in the range of frequencies $[\nu,\nu + d\nu]$ and is given by the expression
\begin{equation}
\rho(\nu,T)_\mathrm{Classical} d\nu = \frac{8 \pi k_B T}{c^3} \nu^2 d\nu \quad [\si{\joule\times\meter^{-3}}]
\end{equation}
where $k_B$ is the Boltzmann constant
\begin{equation}
k_B = \frac{R}{N_A} = 1.38 \times 10^{-23}\;\si{\joule\per\kelvin} \text{particle}^{-1}.
\end{equation}


\begin{exercise}
Draw a plot of $\rho(\nu,T)_\mathrm{Classical}$ for two temperatures, $T_1$ and $T_2$ with $T_1 < T_2$.
\end{exercise}
\begin{exercise}
The total energy emitted by a blackbody is given by the integral over all emitted frequencies, that is, the integral over $\nu$ from 0 to infinity of $\rho(\nu,T)_\mathrm{Classical}$.
Evaluate this integral.
\begin{equation}
\int_0^\infty \rho(\nu,T)_\mathrm{Classical} d\nu 
= \frac{8 \pi k_B T}{c^3} \int_0^\infty  \nu^2 d\nu
= \frac{8 \pi k_B T}{c^3} \left.  \frac{\nu^3}{3}  \right|_0^\infty = \infty !
\end{equation}
\end{exercise}
%\begin{example}
%Hello
%\end{example}

The divergence of the integral that determines the total energy emitted is known as the \textbf{ultraviolet catastrophe} because it is due to the incorrect behavior of $\rho(\nu,T)_\mathrm{Classical}$ for large frequencies.
The solution to this problem came from Max Planck (1900).
Planck assumed that light could be describe as a set of harmonic oscillators, and that each oscillator carried only a discrete amount of energy:
\begin{equation}
E = n h \nu,
\end{equation}
where $n = 0, 1, 2, \ldots$ is a positive integer and $h$ is Planck's constant
\begin{equation}
h = 6.626070040(81) 10^{-34}\;\si{\joule\second}.
\end{equation}
Using this assumption Planck derived the following equation
\begin{equation}
\rho(\nu,T)_\mathrm{Planck} =  \frac{8 \pi h}{c^3} \frac{\nu^3 d\nu}{e^{h\nu/k_B T} -1},
\end{equation}
which shows the correct behavior.

\begin{exercise}
What is the value of $\rho(\nu,T)_\mathrm{Planck}$ for $\nu = 0$?
Expand $\rho(\nu,T)_\mathrm{Planck}$ as a Taylor series in the variable $\nu$ centered around $\nu = 0$.
Does this equation agree with the classical expression $\rho(\nu,T)_\mathrm{Classical}$?
\end{exercise}

\subsection{The photoelectric effect}
The photoelectric effect, discovered by Heinrich Hertz (1887), consists in the production of electric sparks when a metal is illuminated with ultraviolet light.
Experimentally, one finds that the kinetic energy ($K$) of the electrons emitted is given by
\begin{equation}
K = \begin{cases}
0 & \text{if } h\nu - \Phi  \leq 0 \\
h\nu - \Phi & \text{if } h\nu - \Phi  > 0,
\end{cases}
\end{equation}
where $\Phi$ is called the work function and depends on the metal and $h$ is Planck's constant.
The work function is the amount of energy required to remove an electron from the metal and create a free electron with zero kinetic energy.
This result disagrees with a classical interpretation of the experiment, in which one assumes that the kinetic energy is proportional to the intensity of the light used in the experiment.

Albert Einstein was able to explain the photoelectric effect by assuming that a photon of frequency $\nu$ carries an energy equal to
\begin{equation}
E = h \nu,
\end{equation}
and that each photon can produce only one electron.
Using the relationship between frequency and wavelength we also can express the energy of a photon as
\begin{equation}
E = h \frac{c}{\lambda}.
\end{equation}

\subsection{Units and conversion factors}
Spectroscopist use a variety of units to report energy differences.
The wave number ($\tilde{\nu}$) is defined as the number of waves per unit length
\begin{equation}
\tilde{\nu} = \frac{1}{\lambda} = \frac{E}{hc}.
\end{equation}
so that in SI units it is measured in $\si{\per\meter}$. However, it is more convenient to express wave number in units of $\si{\per\centi\meter}$, which is done by multiplying by 0.01.
Here are some conversion factors to remember:
\begin{equation}
\SI{1}{kcal\per\mole} = \SI{349.75}{\per\centi\meter}\,\mathrm{molecule}^{-1},
\end{equation}
and 
\begin{equation}
\SI{1}{\per\centi\meter}\,\mathrm{molecule}^{-1} = \SI{29979.2458}{\mega\hertz}.
\end{equation}
Another useful energy unit is electron volt (eV)
\begin{equation}
\SI{1}{\electronvolt} = (\text{charge of }e^{-})(1\si{\volt})  =  1.602\times10^{-19} \underbrace{C V}_{J} 
=\SI{8065.54}{\per\centi\meter}.
\end{equation}

\subsection{Time scales for molecular motion}

\begin{table}[htbp]
\centering
\caption{Regions of the Electromagnetic Spectrum} % requires the topcapt package
   \begin{tabular}{@{} lllll @{}} % Column formatting, @{} suppresses leading/trailing space
      \toprule
      Region & Wavelength & Wave number & Frequency & Molecular process \\
      \midrule
      Microwave & 0.04--25 \si{\centi\meter} & 0.04--25 \si{\per\centi\meter} & 1.2--750 \si{\giga\Hz} & Molecular rotations \\
Infrared & 2.5--25 \si{\micro\meter} & 400--4000  \si{\per\centi\meter} & $1.2 \times 10^{13}$--$1.2 \times 10^{14}$ \si{\Hz} & Fundamental vibrations \\
Visible & 400--800 \si{\nano\meter} & 12500--25000  \si{\per\centi\meter} & $3.7 \times 10^{14}$--$7.5 \times 10^{14}$ \si{\Hz} & Valence electronic excitations \\
Near ultraviolet & 10--200 \si{\nano\meter} & 25000--50000  \si{\per\centi\meter} & $7.5 \times 10^{14}$--$1.5 \times 10^{15}$ \si{\Hz} & Valence electronic excitations \\
X-rays & 1--10 \si{\angstrom} & $10^7$--$10^8$   \si{\per\centi\meter} & $3 \times 10^{17}$--$3 \times 10^{18}$ \si{\Hz} & Ionization of inner shell electrons \\
      \bottomrule
   \end{tabular}
   \label{tab:electromagnetic_spectrum}
\end{table}

Regions of the electromagnetic spectrum with the corresponding molecular transitions are reported in Table~\ref{tab:electromagnetic_spectrum}.

We can estimate time scales ($\tau$) as the inverse of the frequency
\begin{equation}
\tau = \frac{1}{\nu} = \frac{\lambda}{c} = \frac{1}{\tilde{\nu}{c}}
= \frac{1}{ (\tilde{\nu} \, \si{\per\cm}) \times (2.99 \times 10^{10}\si{\cm\per\second}) }
\end{equation}

Rotations
\begin{equation}
\tilde{\nu} = \SI{1}{\per\centi\meter} \Rightarrow \tau_\mathrm{rot} = 3.3 \times 10^{-11} \si{\per\second} = \SI{33}{\pico\second}
\end{equation}

Vibrations
\begin{equation}
\tilde{\nu} = \SI{2000}{\per\centi\meter} \Rightarrow \tau_\mathrm{vib} = 1.7 \times 10^{-14} \si{\per\second}
= \SI{17}{\femto\second}
\end{equation}

Electronic orbits
\begin{equation}
\tilde{\nu} = \SI{40000}{\per\centi\meter} \Rightarrow \tau_\mathrm{el} = 8.3 \times 10^{-16} \si{\per\second}
= \SI{830}{\atto\second}
\end{equation}



%\begin{example}
%Hello
%\end{example}






