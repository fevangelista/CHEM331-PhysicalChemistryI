% This work is licensed under the Creative Commons Attribution-NonCommercial 4.0 International License.
% To view a copy of this license, visit http://creativecommons.org/licenses/by-nc/4.0/
% or send a letter to Creative Commons, PO Box 1866, Mountain View, CA 94042, USA.

% !TEX TS-program = xelatex

\documentclass[../Main/chem331-notes.tex]{subfiles}
\begin{document}

\setcounter{section}{9}

\section{Vibrations in diatomic molecules}
\subsection{Separation of internal and center of mass motion}
In this section we will study how to apply the harmonic oscillator model to a diatomic molecule.
The first aspect we have to study is how to connect the Hamiltonian of a diatomic molecule to that of the harmonic oscillator.
For a diatomic molecule the Hamiltonian will depend on the coordinate of atom 1 ($x_1$) and atom 2 ($x_2$), as well as, their corresponding masses $m_1$ and $m_2$.
The kinetic energy is
\begin{equation}
\hat{T} = -\frac{\hbar^2}{2 m_1} \frac{\partial^2}{\partial x_1^2} + -\frac{\hbar^2}{2 m_2} \frac{\partial^2}{\partial x_2^2},
\end{equation}
while the potential is
\begin{equation}
\hat{V} = \frac{1}{2} k (x_2 - x_1 - r_e)^2,
\end{equation}
where we assumed that the potential is $(x_2 - x_1 - r_e)^2$ is the square of the distance of the two atoms.
This potential is zero at the equilibrium bond distance, when $x_2 - x_1 = r_e$, otherwise it is positive.
Now we will introduce two new variables, the bond distance ($r$) and the center of mass ($R$), defined as
\begin{align}
r & = x_2 - x_1 \\
R & = \frac{m_1 x_1 + m_2 x_2}{m_1 + m_2} = \frac{m_1 x_1 + m_2 x_2}{M},
\end{align}
where $M = m_1 + m_2$ is the total mass.

In this new system of coordinates the potential simplifies to
\begin{equation}
\hat{V} = \frac{1}{2} k (x_2 - x_1 - r_e)^2 = \frac{1}{2} k (r - r_e)^2,
\end{equation}

Applying the chain rule we express the derivatives in $x_1$ and $x_2$ as derivatives in $r$ and $R$
\begin{equation}
\frac{\partial}{\partial x_1} = \frac{\partial r}{\partial x_1} \frac{\partial}{\partial r} + \frac{\partial r}{\partial x_1} \frac{\partial}{\partial r},
\end{equation}
where
\begin{equation}
\frac{\partial r}{\partial x_1} = \frac{\partial}{\partial x_1} \left( x_2 - x_1 \right) = -1,
\end{equation}
and
\begin{equation}
\frac{\partial R}{\partial x_1} = \frac{\partial}{\partial x_1} \left(\frac{m_1 x_1 + m_2 x_2}{m_1 + m_2} \right) = \frac{m_1}{M}.
\end{equation}
So we can write
\begin{equation}
\frac{\partial}{\partial x_1} = - \frac{\partial}{\partial r} + \frac{m_1}{M}\frac{\partial}{\partial R},
\end{equation}
and the second derivative is
\begin{equation}
\frac{\partial^2}{\partial x^2_1} = \frac{\partial^2}{\partial r^2} + \left(\frac{m_1}{M}\right)^2\frac{\partial^2}{\partial R^2} - \frac{2 m_1}{M}\frac{\partial^2}{\partial R \partial r}.
\end{equation}
Repeating the same for $x_2$ we get
\begin{equation}
\frac{\partial^2}{\partial x^2_2} = \frac{\partial^2}{\partial r^2} + \left(\frac{m_2}{M}\right)^2\frac{\partial^2}{\partial R^2} + \frac{2 m_2}{M}\frac{\partial^2}{\partial R \partial r}.
\end{equation}

Plugging these two derivatives in the kinetic energy operator we get
\begin{align}
\hat{T} = & -\frac{\hbar^2}{2 m_1} \left[
\frac{\partial^2}{\partial r^2} + \left(\frac{m_1}{M}\right)^2\frac{\partial^2}{\partial R^2} - \frac{2 m_1}{M}\frac{\partial^2}{\partial R \partial r}
\right] \\
  & -\frac{\hbar^2}{2 m_2} 
   \left[
\frac{\partial^2}{\partial r^2} + \left(\frac{m_2}{M}\right)^2\frac{\partial^2}{\partial R^2} + \frac{2 m_2}{M}\frac{\partial^2}{\partial R \partial r}
   \right] \\
= &  -\frac{\hbar^2}{2} \frac{\partial^2}{\partial R^2} \left(\frac{m_1}{M^2} + \frac{m_2}{M^2}\right)   -\frac{\hbar^2}{2} \frac{\partial^2}{\partial r^2} \left(\frac{1}{m_1} + \frac{1}{m_2}  \right)  \\
= & -\frac{\hbar^2}{2M} \frac{\partial^2}{\partial R^2} -\frac{\hbar^2}{2 \mu} \frac{\partial^2}{\partial r^2} .
\end{align}
The quantity $\mu$ is called the \textbf{reduced mass}
\begin{iequation}
\mu = \frac{1}{m_1} + \frac{1}{m_2} = \frac{m_1 + m_2}{m_1  m_2}.
\end{iequation}
Putting all together we can write the Hamiltonian as
\begin{equation}
\hat{H} =  -\frac{\hbar^2}{2M} \frac{\partial^2}{\partial R^2} -\frac{\hbar^2}{2 \mu} \frac{\partial^2}{\partial r^2} + \frac{1}{2} k (r - r_e)^2 = \hat{H}_\mathrm{CM} + \hat{H}_\mathrm{HO},
\end{equation}
which we separated as a center-of-mass Hamiltonian (CM) and a harmonic oscillator Hamiltonian (HO),
\begin{equation}
\hat{H}_\mathrm{CM} = -\frac{\hbar^2}{2M} \frac{\partial^2}{\partial R^2},
\end{equation}
and
\begin{equation}
\hat{H}_\mathrm{HO} = -\frac{\hbar^2}{2 \mu} \frac{\partial^2}{\partial r^2} + \frac{1}{2} k (r - r_e)^2.
\end{equation}

\begin{example}[Vibrations of diatomics]
The force constant of H${}^{127}$I is $k = 291 \si{\newton\per\meter}$. Calculate the vibrational frequency of H${}^{127}$I in $\si{\per\centi\meter}$.

Solution: The energy difference between two level of the harmonic oscillator is
\begin{equation}
E = \hbar \omega = \hbar \sqrt{\frac{k}{\mu}},
\end{equation}
and the wave numbers corresponding to this transition are
\begin{equation}
\tilde{\nu} = \frac{1}{\lambda} = \frac{\nu}{c} =  \frac{E}{hc} = \frac{1}{2\pi c}\sqrt{\frac{k}{\mu}}.
\end{equation}
The reduced mass of H${}^{127}$I is
\begin{equation}
\mu = \frac{m_1  m_2}{m_1 + m_2} = \frac{1.007825 \times 126.904468}{1.007825 + 126.904468} 1.660539040 \times 10^{-27} \si{\kilogram} = 1.66035 \times 10^{-27} \si{\kilogram}
\end{equation}

Plugging the number in we get
\begin{equation}
\tilde{\nu} = \frac{1}{2 \pi \times \SI{299792458}{\meter\per\second}}\sqrt{\frac{291 \si{\newton\per\meter}}{1.66035 \times 10^{-27} \si{\kilogram}}} = \SI{222252}{\per\meter} = \SI{2222.52}{\per\centi\meter}
\end{equation}

\end{example}





\end{document}

