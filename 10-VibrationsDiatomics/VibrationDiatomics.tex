% This work is licensed under the Creative Commons Attribution-NonCommercial 4.0 International License.
% To view a copy of this license, visit http://creativecommons.org/licenses/by-nc/4.0/
% or send a letter to Creative Commons, PO Box 1866, Mountain View, CA 94042, USA.

% !TEX TS-program = xelatex

\documentclass[../Main/chem331-notes.tex]{subfiles}
\begin{document}

\setcounter{section}{9}

\section{Vibrations in diatomic molecules}
\subsection{Separation of internal and center of mass motion}
In this section we will study how to apply the harmonic oscillator model to a diatomic molecule.
The first aspect we have to study is how to connect the Hamiltonian of a diatomic molecule to that of the harmonic oscillator.
For a diatomic molecule the Hamiltonian will depend on the coordinate of atom 1 ($x_1$) and atom 2 ($x_2$), as well as, their corresponding masses $m_1$ and $m_2$.
The kinetic energy is
\begin{equation}
\hat{T} = -\frac{\hbar^2}{2 m_1} \frac{\partial^2}{\partial x_1^2} + -\frac{\hbar^2}{2 m_2} \frac{\partial^2}{\partial x_2^2},
\end{equation}
while the potential is
\begin{equation}
\hat{V} = \frac{1}{2} k (x_2 - x_1 - r_e)^2,
\end{equation}
where we assumed that the potential is $(x_2 - x_1 - r_e)^2$ is the square of the distance of the two atoms.
This potential is zero at the equilibrium bond distance, when $x_2 - x_1 = r_e$, otherwise it is positive.
Now we will introduce two new variables, the bond distance ($r$) and the center of mass ($R$), defined as
\begin{align}
r & = x_2 - x_1 \\
R & = \frac{m_1 x_1 + m_2 x_2}{m_1 + m_2} = \frac{m_1 x_1 + m_2 x_2}{M},
\end{align}
where $M = m_1 + m_2$ is the total mass.

In this new system of coordinates the potential simplifies to
\begin{equation}
\hat{V} = \frac{1}{2} k (x_2 - x_1 - r_e)^2 = \frac{1}{2} k (r - r_e)^2,
\end{equation}

Applying the chain rule we express the derivatives in $x_1$ and $x_2$ as derivatives in $r$ and $R$
\begin{equation}
\frac{\partial}{\partial x_1} = \frac{\partial r}{\partial x_1} \frac{\partial}{\partial r} + \frac{\partial r}{\partial x_1} \frac{\partial}{\partial r},
\end{equation}
where
\begin{equation}
\frac{\partial r}{\partial x_1} = \frac{\partial}{\partial x_1} \left( x_2 - x_1 \right) = -1,
\end{equation}
and
\begin{equation}
\frac{\partial R}{\partial x_1} = \frac{\partial}{\partial x_1} \left(\frac{m_1 x_1 + m_2 x_2}{m_1 + m_2} \right) = \frac{m_1}{M}.
\end{equation}
So we can write
\begin{equation}
\frac{\partial}{\partial x_1} = - \frac{\partial}{\partial r} + \frac{m_1}{M}\frac{\partial}{\partial R},
\end{equation}
and the second derivative is
\begin{equation}
\frac{\partial^2}{\partial x^2_1} = \frac{\partial^2}{\partial r^2} + \left(\frac{m_1}{M}\right)^2\frac{\partial^2}{\partial R^2} - \frac{2 m_1}{M}\frac{\partial^2}{\partial R \partial r}.
\end{equation}
Repeating the same for $x_2$ we get
\begin{equation}
\frac{\partial^2}{\partial x^2_2} = \frac{\partial^2}{\partial r^2} + \left(\frac{m_2}{M}\right)^2\frac{\partial^2}{\partial R^2} + \frac{2 m_2}{M}\frac{\partial^2}{\partial R \partial r}.
\end{equation}

Plugging these two derivatives in the kinetic energy operator we get
\begin{align}
\hat{T} = & -\frac{\hbar^2}{2 m_1} \left[
\frac{\partial^2}{\partial r^2} + \left(\frac{m_1}{M}\right)^2\frac{\partial^2}{\partial R^2} - \frac{2 m_1}{M}\frac{\partial^2}{\partial R \partial r}
\right] \\
  & -\frac{\hbar^2}{2 m_2} 
   \left[
\frac{\partial^2}{\partial r^2} + \left(\frac{m_2}{M}\right)^2\frac{\partial^2}{\partial R^2} + \frac{2 m_2}{M}\frac{\partial^2}{\partial R \partial r}
   \right] \\
= &  -\frac{\hbar^2}{2} \frac{\partial^2}{\partial R^2} \left(\frac{m_1}{M^2} + \frac{m_2}{M^2}\right)   -\frac{\hbar^2}{2} \frac{\partial^2}{\partial r^2} \left(\frac{1}{m_1} + \frac{1}{m_2}  \right)  \\
= & -\frac{\hbar^2}{2M} \frac{\partial^2}{\partial R^2} -\frac{\hbar^2}{2 \mu} \frac{\partial^2}{\partial r^2} .
\end{align}
The quantity $\mu$ is called the \textbf{reduced mass}
\begin{iequation}
\mu = \frac{1}{m_1} + \frac{1}{m_2} = \frac{m_1 + m_2}{m_1  m_2}.
\end{iequation}
Putting all together we can write the Hamiltonian as
\begin{equation}
\hat{H} =  -\frac{\hbar^2}{2M} \frac{\partial^2}{\partial R^2} -\frac{\hbar^2}{2 \mu} \frac{\partial^2}{\partial r^2} + \frac{1}{2} k (r - r_e)^2 = \hat{H}_\mathrm{CM} + \hat{H}_\mathrm{HO},
\end{equation}
which we separated as a center-of-mass Hamiltonian (CM) and a harmonic oscillator Hamiltonian (HO),
\begin{equation}
\hat{H}_\mathrm{CM} = -\frac{\hbar^2}{2M} \frac{\partial^2}{\partial R^2},
\end{equation}
and
\begin{equation}
\hat{H}_\mathrm{HO} = -\frac{\hbar^2}{2 \mu} \frac{\partial^2}{\partial r^2} + \frac{1}{2} k (r - r_e)^2.
\end{equation}

Recall that the center-of-mass Hamiltonian is the same as that for a particle in a box. So we can solve this problem for a particle in a box of dimension $L$ much larger than the molecule bond length. 

Note that the Hamiltonian can be written as a sum of two separate terms that act on different coordinates.
The center-of-mass Hamiltonian ($\hat{H}_\mathrm{CM}$) depends only on $R$, while the harmonic oscillator Hamiltonian ($\hat{H}_\mathrm{HO}$) acts only on $r$.
In this case we say that the Hamiltonian is \textbf{separable}, and the eigenvalues and eigenfunctions are related to those of the two separate Hamiltonians.\mnote{In general, if the Hamiltonian splits into two parts $\hat{H} = \hat{H}_A + \hat{H}_B$, where $\hat{H}_A$ and $\hat{H}_B$ act on different variables $x$ and $y$, then the eigenvalues and eigenfunctions of $\hat{H}$ are $E = E_A + E_B$ and $\psi(x,y) = \psi_A(x) \psi_B(y)$, where $\hat{H}_A\psi_A(x) = E_A \psi_A(x)$ and $\hat{H}_B\psi_B(x) = E_B \psi_B(x)$.}
The eigenvalues are the sum of the eigenvalues for a particle in a box (of mass $M$ and length $L$) and a harmonic oscillator (with $\omega = \sqrt{\frac{k}{\mu}}$)
\begin{equation}
E_{n,v} = E_{n}^\mathrm{CM} +  E_{v}^\mathrm{HO} = \frac{h^2 n^2}{8 M L^2} +  \hbar \omega \left( v + \frac{1}{2}\right)
\end{equation}

The eigenfunctions are instead given by
\begin{equation}
\psi(R,r) = \psi^\mathrm{PIB}_n(R) \psi^\mathrm{HO}_v(r),
\end{equation}
where $\psi^\mathrm{PIB}_n(R)$ and $\psi^\mathrm{HO}_v(r)$ are the eigenfunctions for a particle in a box and a harmonic oscillator, respectively.

\subsection{Selection rule and vibrational spectra}
If we are interested in the vibrational spectroscopy of diatomic molecules we can focus exclusively on the $v$ dependence of the energy.

Not all transitions between the energy levels of the harmonic oscillator are allowed. The \textbf{selection rule}\mnote{Selection rules can be justified with quantum mechanics, but this subject is a bit advanced and will be covered later} for a harmonic oscillator tells us that the only allowed transitions are those that satisfy
\begin{equation}
\Delta v = \pm 1.
\end{equation}
Absorption of a photon can promote a harmonic oscillator from the level $v$ to $v+1$. The energy of the photon corresponding to this transition is given by the energy difference of these two levels
\begin{equation}
\Delta E_{v}= E_{v+1} - E_{v} =\hbar \omega \left( v + 1 + \frac{1}{2}\right) - \hbar \omega \left( v + \frac{1}{2}\right)
= \hbar \omega,
\end{equation}
where
\begin{equation}
\omega = \hbar \sqrt{\frac{k}{\mu}}.
\end{equation}
Note that this difference is independent of $v$, so all transitions of a harmonic oscillator occur at the same frequency.

\begin{example}[Vibrations of diatomics]
The force constant of H${}^{127}$I is $k = 291 \si{\newton\per\meter}$. Calculate the vibrational frequency of H${}^{127}$I in $\si{\per\centi\meter}$.

Solution: The energy difference between two level of the harmonic oscillator is
\begin{equation}
E = \hbar \omega = \hbar \sqrt{\frac{k}{\mu}},
\end{equation}
and the wave numbers corresponding to this transition are
\begin{equation}
\tilde{\nu} = \frac{1}{\lambda} = \frac{\nu}{c} =  \frac{E}{hc} = \frac{1}{2\pi c}\sqrt{\frac{k}{\mu}}.
\end{equation}
The reduced mass of H${}^{127}$I is
\begin{equation}
\mu = \frac{m_1  m_2}{m_1 + m_2} = \frac{1.007825 \times 126.904468}{1.007825 + 126.904468} 1.660539040 \times 10^{-27} \si{\kilogram} = 1.66035 \times 10^{-27} \si{\kilogram}
\end{equation}

Plugging the number in we get
\begin{equation}
\tilde{\nu} = \frac{1}{2 \pi \times \SI{299792458}{\meter\per\second}}\sqrt{\frac{291 \si{\newton\per\meter}}{1.66035 \times 10^{-27} \si{\kilogram}}} = \SI{222252}{\per\meter} = \SI{2222.52}{\per\centi\meter}
\end{equation}
\end{example}



\subsection{The Morse potential}
A convenient approximation to the potential of a diatomic molecule is the Morse potential
\begin{equation}
V(r) = D \left[1 - e^{-\beta (r - r_e)} \right]^2.
\end{equation}
This potential contains two parameters. The first one, $D$, is the dissociation energy of the molecule, while $r_e$ is the equilibrium bond distance. Both  $D$ and $\beta$ are assumed to be positive.

To find the equilibrium geometry for the Morse potential we take the first derivative and set it to zero
\begin{equation}
\frac{d V(r)}{dr} = 2 D \left[1 - e^{-\beta (r - r_e)} \right] \frac{d}{dr} \left[1 - e^{-\beta (r - r_e)} \right]
= 2 D \beta \left[1 - e^{-\beta (r - r_e)} \right] e^{-\beta (r - r_e)}.
\end{equation}
Setting the first derivative to zero requires solving the following equation
\begin{equation}
2 D \beta \left[1 - e^{-\beta (r - r_e)} \right] e^{-\beta (r - r_e)} = 0
\end{equation}
This expression is zero when $r = r_e$. To see this substitute $r$ with $r_e$ in the term in square brackets
\begin{equation}
1 - e^{-\beta (r_e - r_e)}  = 1 - e^{0} = 1 - 1 = 0.
\end{equation}

Once we know the equilibrium bond distance we can compute the dissociation energy ($D_e$) of the Morse potential.
This quantity is defined as
\begin{equation}
D_e = V(\infty) - V(r_e).
\end{equation}
The first term is equal to
\begin{equation}
V(\infty) = \lim_{r \rightarrow \infty} D \left[1 - e^{-\beta (r - r_e)} \right]^2 = D,
\end{equation}
while the second one is equal to
\begin{equation}
V(r_e) = D \left[1 - e^{-\beta (r_e - r_e)} \right]^2 = 0.
\end{equation}
So we find that for the Morse potential the dissociation energy is equal to the constant $D$
\begin{equation}
D_e = D.
\end{equation}

The second derivative of the Morse potential is given by
\begin{equation}
\begin{split}
\frac{d^2 V(r)}{dr^2} = &  2 D \beta \frac{d}{dr} \left[1 - e^{-\beta (r - r_e)} \right] e^{-\beta (r - r_e)} \\
= & 2 D \beta \left[ \beta e^{-2\beta (r - r_e)}  - \beta  \left[1 - e^{-\beta (r - r_e)} \right] e^{-\beta (r - r_e)}\right] \\
= & 2 D \beta^2  \left[ 2e^{-2 \beta (r - r_e)} - e^{-\beta (r - r_e)}\right].
\end{split}
\end{equation}
The force constant for the Morse potential is the second derivative evaluated at the equilibrium bond distance, $r = r_e$
\begin{equation}
k = \left. \frac{d^2 V(r)}{dr^2} \right|_{r = r_e} =  2 D \beta^2.
\end{equation}



\end{document}

