% This work is licensed under the Creative Commons Attribution-NonCommercial 4.0 International License.
% To view a copy of this license, visit http://creativecommons.org/licenses/by-nc/4.0/
% or send a letter to Creative Commons, PO Box 1866, Mountain View, CA 94042, USA.

% !TEX TS-program = xelatex

\documentclass[../Main/chem331-notes.tex]{subfiles}
\begin{document}

\setcounter{section}{4}

\section{Operators and eigenvalue equations}
\subsection{Generalities}
Operators are mathematical objects that transform functions.
We have already encountered one operator, the Hamiltonian for a particle in one dimension
\begin{equation}
\hat{H} =  - \frac{\hbar^2}{2 m} \frac{\partial^2}{\partial x^2} + V(x).
\end{equation}
This contains two terms, a second derivative and a multiplicative function.

Operators play an important role in quantum mechanics because for each classical physical property (observable) $O$ there is a corresponding quantum mechanical operator $\hat{O}$ that is \textbf{linear and Hermitian}
\begin{equation}
(\text{observable}) \; O \rightarrow \hat{O} \; (\text{operator}).
\end{equation}
An operator is said to be linear if for any function $f(x)$ and $g(x)$, it satisfies the condition
\begin{equation}
\hat{O}[c_1 f(x) + c_2 g(x)] = c_1 \hat{O} f(x) +  c_2 \hat{O} g(x).
\end{equation}
In other words, if an operator is linear we can apply to each term in a sum separately.
The concept of Hermiticity is a bit more involved and we will introduce it later. For now we will take a look at various operators that we will encounter often in this course.

One of the simplest operator \textbf{position operator} $\hat{x}$ which corresponds to the $x$-coordinate of a particle in one dimension
\begin{iequation}
\hat{x} = x ,
\end{iequation}
In this case the operator just multiplies the quantity onto which it is applied by the number $x$.

The \textbf{momentum operator} $\hat{p}$ is related to the classical momentum $p$ and is given by
\begin{iequation}
\hat{p} = -i \hbar \frac{\partial}{\partial x}.
\end{iequation}
Admittedly, this definition is a bit odd and does not resemble the classical momentum operator.

We have already encountered the \textbf{kinetic operator} $\hat{T}$, but we can now show that is related to the momentum operator from the definition of the kinetic energy. Classically we have that
\begin{equation}
T = \frac{1}{2} m v^2 = \frac{1}{2 m} p^2,
\end{equation}
and this expression translates to the case of quantum mechanics as
\begin{equation}
\hat{T} =  \frac{1}{2 m} \hat{p}^2 = \frac{1}{2 m} \left(  -i \hbar \frac{\partial}{\partial x} \right)^2 
= -\frac{\hbar^2}{2m} \frac{\partial^2}{\partial x^2}.
\end{equation}

\subsection{Eigenvalue equations}
As we have seen before, the time-independent Schr\"{o}dinger equation is
\begin{equation}
\hat{H}\psi(x) = E \psi(x).
\end{equation}
This is a special type of equation because it tells us that when we apply the Hamiltonian operator onto the wave function we get back the wave function multiplied by a constant ($E$).
Equations of this type are called eigenvalue equations and play a very important role in quantum mechanics.
A general eigenvalue equation can be written as
\begin{equation}
\hat{O} f(x) = \lambda f(x),
\end{equation}
where $f(x)$ is called the \textbf{eigenfunction} of the operator $\hat{O}$ and $\lambda$ is the corresponding \textbf{eigenvalue}.
%Note that eigenvalues and eigenfunctions form a pair
As an example, consider the operator $\hat{O} = \frac{d}{dx}$.
It is easy to verify that the function $f(x) = e^{\alpha x}$ is an eigenfunction of $\hat{O}$. Applying the operator to the function [$\hat{O} f(x)$] we see that the result is equal to a constant times $f(x)$
\begin{equation}
\hat{O} f(x) = \frac{d}{dx} e^{\alpha x} = \alpha e^{\alpha x} = \alpha f(x).
\end{equation}
In this case the eigenvalue corresponding to $f(x)$ is $\alpha$.

It is important to be aware that eigenvalues may be degenerate, that is, there may be two or more eigenstates of a given operator with the same eigenvalues.
For example, consider the kinetic energy operator $\hat{T} = - \frac{\hbar^2}{2m} \frac{\partial^2}{\partial x^2}$ and the wave functions $\psi_1(x) = e^{ikx}$ and $\psi_2(x) = e^{-ikx}$.
These functions differ by more than a phase factor\mnote{That is we cannot find a complex number $e^{i\theta}$ such that $\psi_1(x) = e^{i\theta} \psi_2(x)$.} and thus from the point of view of quantum mechanics represent two different states of a system.
However, it is easy to verify that both states are eigenfunctions of the kinetic operator with the same eigenvalue
\begin{equation}
\hat{T} \psi_1(x) = - \frac{\hbar^2}{2m} \frac{\partial^2}{\partial x^2} e^{ikx}
 = \frac{\hbar^2 k^2}{2m} e^{ikx},
\end{equation}
 \begin{equation}
\hat{T} \psi_2(x) = - \frac{\hbar^2}{2m} \frac{\partial^2}{\partial x^2} e^{-ikx}
 = \frac{\hbar^2 k^2}{2m} e^{ikx}.
\end{equation}


\subsection{Operators, measurement, and expectation values}
The eigenvalues of an operator play a very important part in quantum mechanics.
Suppose that we are interested in measuring an observable $O$ to which corresponds the operator $\hat{O}$.
Quantum mechanics postulates that the outcome of such a measurement is going to be an eigenvalue of $\hat{O}$ and the final state after the measurement is the eigenstate corresponding to it.\mnote{Here we assume that the eigenvalues of $\hat{O}$ are not degenerate.}
Take for example the Hamiltonian of a particle in a potential $V(x)$ and assume that we had a way to determine all of its eigenvalues
\begin{equation}
\hat{H} \psi_n(x) = E_n \psi_n(x),
\end{equation}
where $n$ is an integer that labels the eigenvalues/eigenfunction pairs in order of increasing energy ($E_1 < E_2 < E_3 < \ldots$).
Then if we perform a measurement of the energy the outcomes will be one of the eigenvalues of $\hat{H}$, say $E_i$.
We can also say that after the measurement, the system wave function will be $\psi_i(x)$, which is the eigenfunction corresponding to $E_i$.
Since the eigenvalues of the Hamiltonian can tell us which energy levels are allowed, they are one of the most important piece of information we can extract from $\hat{H}$.


\subsection{The order of operators is important!}
When we manipulate expressions that contain operators we have to keep in mind that the order in which operators are written does matter.
For any pair of numbers, either real or complex, it does not matter in which order we take the product. So if $x$ and $y$ are numbers we have that
\begin{equation}
x y = y x.
\end{equation}
With commutators this is not generally true.
Let us first define what we mean by a product of operators. If we have two operators, $\hat{A}$ and $\hat{B}$, then the product $\hat{A}\hat{B}$ is the result of applying first the operator $\hat{B}$ and then $\hat{A}$. When the product $\hat{A}\hat{B}$ is applied to a function $f(x)$ we evaluate the result as follows
\begin{equation}
\hat{A}\hat{B} f(x) = \hat{A} \left[\hat{B} f(x)\right]  
\end{equation}

In general for operators the commutative law does not apply
\begin{equation}
\hat{A}\hat{B} \neq \hat{B}\hat{A}.
\end{equation}
To see why the order of operators is important we consider the product of the position and momentum operators. If we apply $\hat{x} \hat{p}$ to a generic function $f(x)$ we get
\begin{equation}
\hat{x} \hat{p} f(x) = x \left(-i\hbar \frac{\partial}{\partial x} f(x) \right)
= -i \hbar x \frac{\partial f(x)}{\partial x}.
\end{equation}
If we instead apply the same operators with the order switched ($\hat{p} \hat{x}$) we get
\begin{equation}
\hat{p} \hat{x} f(x) = \left(-i\hbar \frac{\partial}{\partial x}\right) \left(x  f(x) \right)
= -i \hbar x \frac{\partial f(x)}{\partial x} -i \hbar f(x).
\end{equation}
The two results are clearly not the same.
We are often interested in the difference $\hat{A}\hat{B} - \hat{B}\hat{A}$, which tells us if two operators commute. Because this quantity will appear several times we define a new symbol for it, the commutator, defined as
\begin{equation}
[\hat{A},\hat{B}] = \hat{A}\hat{B} - \hat{B}\hat{A}.
\end{equation}
In the case of $\hat{x}$ and $\hat{p}$ the commutator can be easily found by applying the commutator to a generic function
\begin{equation}
[\hat{x},\hat{p}] f(x) = \hat{x} \hat{p} f(x) - \hat{p} \hat{x} f(x) = i \hbar f(x).
\end{equation}
Since $f(x)$ is general, we conclude that $[\hat{x},\hat{p}] = i \hbar$.
As we will see later, this commutator is related to the uncertainty principle and to whether or not it is possible to measure position and momentum simultaneously.

\subsection{Hermitian operators}
An operator $\hat{A}$ is said to be Hermitian if for every pair of functions $f(x)$ and $g(x)$ it satisfies the integral condition
\begin{equation}
\int_\text{all space} f^*(x) \hat{A} g(x) \, dx = \int_\text{all space} [\hat{A} f(x)]^*  g(x) \, dx.
\end{equation}
Note that if we manipulate the r.h.s we can write this also as
\begin{equation}
\int_\text{all space} f^*(x) \hat{A} g(x) \, dx = \left[\int_\text{all space} g(x)^* \hat{A} f(x)   \, dx \right]^*.
\end{equation}
Hermitian operators are somewhat magic, because when we compute an integral like the one above we can apply them to either functions $f$ or $g$.

Hermitian operators are even more special, because their eigenvalues and eigenfunctions satisfy special properties
\begin{ibox}
\begin{myitems}
\item The eigenvalues of Hermitian operators are \textbf{real}. This is important because in quantum mechanics eigenvalues correspond to the results of measuring the value of a physical observable. This condition then guarantees that all observables are real. Imagine if we measured the position of a particle and quantum mechanics told us that it could be a complex number, how would we interpret that result?
\item The eigenvectors of Hermitian operators are \textbf{orthogonal}. Orthogonality is another property of integrals of functions. This conditions says that if $f(x)$ and $g(x)$ are eigenfunctions of $\hat{A}$, then the integral of $f^*(x)g(x)$ is zero
\begin{equation}
\int_\text{all space} f^*(x) g(x) \, dx = 0.
\end{equation}
Note that orthogonality of functions is a concept related to the orthogonality of vectors. Recall that two vectors $\mathbf{a} = (a_x, a_y, a_z)$ and $\mathbf{b} = (b_x, b_y, b_z)$ are said to be orthogonal if their dot product is zero
\begin{equation}
\mathbf{a} \cdot \mathbf{b} = a_x b_x + a_y b_y + a_z b_z = 0.
\end{equation}
Now we could write this condition as a sum over product of component of $\mathbf{a}$ and $\mathbf{b}$
\begin{equation}
\sum_{i = 1}^{3} a_i b_i = 0,
\end{equation}
and we can similarly rewrite the orthogonality condition as a sum of products of component of a vector
\begin{equation}
\int_a^b f^*(x) g(x) \, dx = \lim_{\Delta x \rightarrow 0} \sum_{k = -\infty}^{\infty} f^*(x_k) g(x_k) \Delta x
=\lim_{\Delta x \rightarrow 0}  \Delta x \sum_{k = -\infty}^{\infty} f^*_k g_k,
\end{equation}
where the the points $x_k$ form a uniform grid with spacing $\Delta x$, $x_k = k  \Delta x$.

\end{myitems}
\end{ibox}

\begin{exercise}
How is orthogonality of wave functions connect to their nodal structure?
\end{exercise}
\begin{exercise}
Show that the momentum operator is Hermitian.
\end{exercise}
To prove that the momentum operator is Hermitian we have to show that
\begin{equation}
\label{eq:operators:momentum}
\int_\text{all space} f^*(x) \hat{p} g(x) \, dx = \int_\text{all space} [\hat{p} f(x)]^*  g(x) \, dx.
\end{equation}
Expand the momentum operator
\begin{equation}
\int_\text{all space} f^*(x) \hat{p} g(x) \, dx =
\int_\text{all space} f^*(x) \left(-i \hbar \frac{d}{dx}\right) g(x) \, dx 
= -i \hbar \int_\text{all space} f^*(x)  \frac{d g(x)}{dx}  \, dx .
\end{equation}
We can now apply integration by parts to the last integral
\begin{equation}
-i \hbar \int_\text{all space} f^*(x)  \frac{d g(x)}{dx}  \, dx 
= -i \hbar \left. f^*(x) g(x)\right|_\text{boundary}  \, dx  +i \hbar \int_\text{all space}  \frac{d f^*(x)}{dx} g(x)  \, dx .
\end{equation}
The first term on the r.h.s. is the product $f^*(x) g(x)$ evaluated at the boundary, which is zero.\mnote{The proof of this is a bit involved and for now we will just take it as a fact that the wave function is either null at the boundary or this term is zero because the wave function is periodic.}
We can manipulate the last term further and write it as
\begin{equation}
+i \hbar \int_\text{all space}  \frac{d f^*(x)}{dx} g(x)  \, dx
=  \int_\text{all space}  \frac{i \hbar d f^*(x)}{dx} g(x)  \, dx
=  \int_\text{all space}  \left[\frac{-i \hbar d f(x)}{dx}\right]^* g(x)  \, dx,
\end{equation}
which is just the r.h.s. of Eq.~\eqref{eq:operators:momentum}.
\end{document}