% !TEX TS-program = xelatex
%!TEX root =  ../3dbox_standalone.tex

\section{The particle in a three-dimensional box}
\subsection{The Schr\"{o}dinger equation in three dimensions}
How do we generalize the Schr\"{o}dinger equation to more than one dimension, say in three dimensions?
A point in three dimensions is specified by a vector $\mathbf{r} = (x,y,z)$.
Wave functions for particles that move in three dimensions will then have to depend on all of these three variables
\begin{equation}
\psi(x) \rightarrow \psi(x,y,z).
\end{equation}
The wave function $\psi(x,y,z)$ is now a complex function that depends on multiple variables.
As we will see, this introduces several complications that will require some advanced mathematical techniques.

Operators for a particle in three dimensions will also become vectors.
For example, the position operator $\hat{x}$ becomes
\begin{equation}
\hat{x} \rightarrow \hat{\mathbf{r}} = (\hat{x}, \hat{y}, \hat{z}),
\end{equation}
where $\hat{x}$, $\hat{y}$, and $\hat{z}$ are position operators for the x, y, and z axes, respectively.
Similarly, the momentum operator will become a vector
\begin{equation}
\hat{p} \rightarrow \hat{\mathbf{p}} =  (\hat{p}_{x}, \hat{p}_{y}, \hat{p}_{z})
= \left(-i\hbar\frac{\partial}{\partial x}, -i\hbar\frac{\partial}{\partial y}, -i\hbar\frac{\partial}{\partial z}\right).
\end{equation}
When we write the kinetic energy operator in three dimensions we have to keep into account that it comes from the square of the momentum operator vector
\begin{equation}
\hat{T} = \frac{\hat{\mathbf{p}}^2}{2m}
= \frac{\hat{\mathbf{p}} \cdot \hat{\mathbf{p}}}{2m} 
= \frac{\hat{p}_{x}^2 + \hat{p}_{y}^2 + \hat{p}_{z}^2}{2m} 
= -\frac{\hbar^2}{2m}
\left(\frac{\partial^2}{\partial x^2} + \frac{\partial^2}{\partial y^2} + \frac{\partial^2}{\partial z^2} \right) = -\frac{\hbar^2}{2m}\nabla^2.
\end{equation}
Here we have expanded the square of the momentum operator as we would do with any vector by summing the square of each component.
In the last term we have written the kinetic operator using the Laplacian operator.\mnote{
The Laplacian is the sum of partial second derivatives, $\nabla^2 = \frac{\partial^2}{\partial x^2} + \frac{\partial^2}{\partial y^2} + \frac{\partial^2}{\partial z^2}$. In other words, the Laplacian is the divergence ($\nabla \cdot$) of the gradient $\nabla$, that is $\nabla^2 = \nabla \cdot \nabla$.}

The total Hamiltonian in three dimension is the sum of the kinetic operator plus a potential operator that depends on all thee coordinates ($\hat{V}(x,y,z)$)
\begin{iequation}
\hat{H} = -\frac{\hbar^2}{2m}
\left(\frac{\partial^2}{\partial x^2} + \frac{\partial^2}{\partial y^2} + \frac{\partial^2}{\partial z^2} \right) + \hat{V}(x,y,z).
\end{iequation}

\subsection{A theorem on separable Hamiltonians}
How can we deal with the Hamiltonian in three dimensions?
Here we will take a look at a method that allows us to simplify the solution of problems in more than one dimension when the Hamiltonian is \textbf{separable}.
Consider the simpler case of a particle in two dimensions $(x,y)$ and suppose that the Hamiltonian is separable into two Hamiltonians that depend on only one of these two variables
\begin{equation}
\hat{H} = \hat{H}_x + \hat{H}_y. 
\end{equation}
In this case, if we know how to solve the Schr\"{o}dinger equation for each individual Hamiltonian $\hat{H}_x$ and $\hat{H}_y$, then we can easily find the solutions to the full Hamiltonian $\hat{H}$.
Suppose that we know the eigenvalues/eigenfunctions of $\hat{H}_x$ and $\hat{H}_y$
\begin{align}
\hat{H}_x \psi_x(x) &= E_x \psi_x(x), \\
\hat{H}_y \psi_y(y) &= E_y \psi_y(y),
\end{align}
then the eigenvalue ($E$) and eigenfunction [$\psi(x,y)$] of the full Hamiltonian $\hat{H}$ are given by
\begin{align}
E &= E_x + E_y, \\
\psi(x,y) &= \psi_x(x) \psi_y(y).
\end{align}

To see that $\psi(x,y) = \psi_x(x) \psi_y(y)$ is an eigenfunction of the full Hamiltonian we plug it in the Schr\"{o}dinger equation
\begin{equation}
\begin{split}
\hat{H} \psi(x,y) =& \hat{H}  \psi_x(x) \psi_y(y) \\
 =& (\hat{H}_x + \hat{H}_y)\psi_x(x) \psi_y(y) \\
=& \hat{H}_x\psi_x(x) \underbrace{\psi_y(y)}_{\text{independent of } x} + \hat{H}_y\psi_x(x) \psi_y(y) \\
=& \psi_y(y) \underbrace{\hat{H}_x\psi_x(x)}_{E_x\psi_x(x)} + \psi_x(x) \hat{H}_y \psi_y(y) \\
=& E_x \psi_y(y)\psi_x(x) + E_y \psi_x(x)  \psi_y(y) \\
=& (E_x + E_y) \psi_x(x)  \psi_y(y) = (E_x + E_y) \psi(x,y).
\end{split}
\end{equation}
In the end we find out that $\psi(x,y) = \psi_x(x) \psi_y(y)$ is indeed an eigenfunction of the Hamiltonian and that the corresponding eigenvalue is $E_x + E_y$.

\subsection{The particle in a three-dimensional box}
Now we are ready to study the case of a particle in a three-dimensional box.
We will assume that the box is rectangular and has edge lengths equal to $L_x$, $L_y$, and $L_z$.
In this case, the potential operator is given by
\begin{equation}
\hat{V}(x,y,z)  =
\begin{cases}
0 & \text{if}\quad 0 \leq x \leq L_x, 0 \leq y \leq L_y, 0 \leq z \leq L_z \\
\infty & \text{otherwise}.
\end{cases}
\end{equation}
The wave function also has to satisfy the following condition
\begin{equation}
\psi(x,y,z) = 0 \quad \text{for } (x,y,z) \text{ outside the box},
\end{equation}
and by demanding that the wave function is continuous we obtain the boundary condition
\begin{equation}
\psi(x,y,z) = 0 \quad \text{for } (x,y,z) \text{ on the borders of the box}.
\end{equation}
When can then focus only on what happens inside the box and write the Hamiltonian
\begin{equation}
\hat{H} = -\frac{\hbar^2}{2m}
\left(\frac{\partial^2}{\partial x^2} + \frac{\partial^2}{\partial y^2} + \frac{\partial^2}{\partial z^2} \right) = \hat{T}_x + \hat{T}_y + \hat{T}_z,
\end{equation}
where $\hat{T}_x$ is the kinetic energy operator for a particle in the $x$ dimension.
Note that this Hamiltonian separates into three independent terms,  $\hat{T}_x$, $\hat{T}_y$, and $\hat{T}_z$, of which we know eigenvalues and eigenfunctions.
In the case of the $x$ coordinate we have the eigenfunctions [$\psi_{n_x}(x)$] and eigenvalues ($E_{n_x}$)
\begin{equation}
\begin{split}
\psi_{n_x}(x) &= \sqrt{\frac{L_x}{2}} \sin \left(\frac{n_x \pi x}{L_x}\right) \\
E_{n_x} &= \frac{h^2 n_x^2}{8mL_x^2},
\end{split}
\end{equation}
where $n_x = 1, 2, \ldots$ is the particle in a box quantum number for the $x$ dimension.
Similar equations can be written for the $y$ and $z$ components of the Hamiltonian.

If we apply the theorem on separable Hamiltonians we get that the total wave function is the product of solutions for the $x$, $y$, and $z$ directions
\begin{iequation}
\begin{split}
\psi_{n_x,n_y,n_z}(x,y,z) =&  \psi_{n_x}(x) \psi_{n_y}(y) \psi_{n_z}(z) \\
=& \sqrt{\frac{8}{L_x L_y L_z}}  \sin \left(\frac{n_x \pi x}{L_x}\right)  \sin \left(\frac{n_y \pi y}{L_y}\right)  \sin \left(\frac{n_z \pi z}{L_z}\right),
\end{split}
\end{iequation}
while the total energy is given by
\begin{iequation}
E_{n_x,n_y,n_z} = E_{n_x} + E_{n_y} + E_{n_z} = \frac{h^2}{8m} \left( 
\frac{n_x^2}{L_x^2} + \frac{n_y^2}{L_y^2} + \frac{n_z^2}{L_z^2} \right).
\end{iequation}

\subsection{Energy levels of a cubic box}
For a cubic box
\begin{equation}
L_x = L_y = L_z = L
\end{equation}
and the energy expression simplifies to
\begin{iequation}
E_{n_x,n_y,n_z} = \frac{h^2}{8mL^2} (n_x^2 + n_y^2 + n_z^2).
\end{iequation}






