% This work is licensed under the Creative Commons Attribution-NonCommercial 4.0 International License.
% To view a copy of this license, visit http://creativecommons.org/licenses/by-nc/4.0/
% or send a letter to Creative Commons, PO Box 1866, Mountain View, CA 94042, USA.

% !TEX TS-program = xelatex

\documentclass[../Main/chem331-notes.tex]{subfiles}
\begin{document}

\setcounter{section}{6}

\section{Expectation value of operators}
\subsection{Average outcome of many measurements}
Quantum mechanics does not allow us to predict the outcome of the measurement of an observable.
For example, if we measure the position of the particle in a box prepared in the eigenstate $\psi_{n}(x)$, we know we will get positions distributes according to the probability distribution
\begin{equation}
\rho(x) = |\psi_{n}(x)|^2 = \frac{2}{L} \sin^2 \left(\frac{n \pi x}{L}\right).
\end{equation}
However, we cannot predict the outcome of each measurement of the position done in an experiment.

One piece of information that we can extract from the wave function is the average outcome of many measurements.
Suppose we have prepared $N$ replicas of the quantum state $\psi(x)$ and on each systems we perform a measurement of the position, recording each measurement, which we denote as $x_k$, with $k = 1, 2, \ldots, N$. Then the average value of $x$---indicated with the symbol $\langle \hat{x} \rangle$---is given by
\begin{equation}
\langle \hat{x} \rangle = \frac{x_1 + x_2 + \ldots + x_N}{N}  =\frac{\sum_{k = 1}^N x_k}{N}.
\end{equation}
It turns out that we can also compute $\langle \hat{x} \rangle$ from the wave function $\psi(x)$ directly as
\begin{equation}
\langle \hat{x} \rangle =  \int_\text{all space} \psi^*(x) \hat{x} \psi(x) \, dx.
\end{equation}
We call this the \textbf{expectation value of the operator} $\hat{x}$.
For a generic operator $\hat{A}$ we write the expectation value as
\begin{iequation}
\langle \hat{A} \rangle =  \int_\text{all space} \psi^*(x) \hat{A} \psi(x) \, dx.
\end{iequation}

For example, let us compute the expectation value of the position operator for an eigenstate $\psi_{n}(x)$ of the particle in a box Hamiltonian. We have to evaluate the integral
\begin{equation}
\langle \hat{x} \rangle
= \int_0^L \sqrt{\frac{2}{L}} \sin \left(\frac{n \pi x}{L}\right) x \sqrt{\frac{2}{L}} \sin \left(\frac{n \pi x}{L}\right) \, dx
= \int_0^L x \frac{2}{L} \sin^2 \left(\frac{n \pi x}{L}\right) \, dx.
\end{equation}
\begin{exercise}
Evaluate $\langle \hat{x} \rangle$ for the particle in a box. 
\end{exercise}

Similarly, the average momentum of a particle in a box is given by
\begin{equation}
\begin{split}
\langle \hat{p} \rangle
&= \int_0^L \sqrt{\frac{2}{L}} \sin \left(\frac{n \pi x}{L}\right) \left( -i \hbar \frac{\partial}{\partial x} \right) \sqrt{\frac{2}{L}} \sin \left(\frac{n \pi x}{L}\right) \, dx \\
&= -i \hbar \frac{2}{L}  \int_0^L \sin \left(\frac{n \pi x}{L}\right) \frac{\partial}{\partial x} \sin \left(\frac{n \pi x}{L}\right)\, dx \\
&= -i \hbar \frac{2}{L} \frac{n \pi}{L}  \int_0^L \sin \left(\frac{n \pi x}{L}\right) \cos \left(\frac{n \pi x}{L}\right)\, dx \\
& = 0.
\end{split}
\end{equation}
To show that the last integral is zero introduce the quantity
\begin{equation}
\theta = \frac{n \pi x}{L},
\end{equation}
which allows us to rewrite\mnote{Here we use the integral $\int \sin(\theta) \cos(\theta)\, d\theta = -\frac{1}{4} \cos(2 \theta) + C$.}
\begin{equation}
\frac{n \pi}{L} \int_0^L \sin \left(\frac{n \pi x}{L}\right) \cos \left(\frac{n \pi x}{L}\right)\, dx 
= \int_0^{n \pi} \sin(\theta) \cos(\theta)\, d\theta 
= \left.-\frac{1}{4} \cos(2 \theta) \right|^{n\pi}_{0}  = 0.
\end{equation}

\subsection{Uncertainty in the outcome of a measurement}
We can further characterize the distribution of measurements in an experiment by calculating the variance or standard deviation of an observable.
For example, the standard deviation for a series of $N$ measurement of the position is computed from statistics as
\begin{equation}
\sigma_x = \sqrt{ \frac{(x_1 - \langle \hat{x} \rangle)^2 + (x_2 - \langle \hat{x} \rangle)^2 +  \ldots + (x_N - \langle \hat{x} \rangle)^2 }{N} } = \sqrt{ \frac{1}{N} \sum_{k = 1}^{N} (x_i - \langle \hat{x} \rangle)^2}
\end{equation}
In quantum mechanics we evaluate this quantity as the expectation value of the operator that measures the deviation of the position from the average $(\hat{x} - \langle \hat{x} \rangle)^2$
\begin{equation}
\sigma^2_x 
=\langle (\hat{x} - \langle \hat{x} \rangle)^2 \rangle =  \int_\text{all space} \psi^*(x) (\hat{x} - \langle \hat{x} \rangle)^2 \psi(x) \, dx.
\end{equation}
Note that in this expression $\langle \hat{x} \rangle$ is just a number.
For a generic operator $\hat{A}$ we write the square of the standard deviation as
\begin{iequation}
\sigma^2_A
=\langle (\hat{A} - \langle \hat{A} \rangle)^2 \rangle
= \int_\text{all space} \psi^*(x) (\hat{A} - \langle \hat{A} \rangle)^2 \psi(x) \, dx.
\end{iequation}
It is not difficult to show that this equation is equivalent to
\begin{equation}
\sigma^2_A
=\langle \hat{A}^2 \rangle - \langle \hat{A} \rangle^2
= \int_\text{all space} \psi^*(x) \hat{A}^2 \psi(x) \, dx
- \left[\int_\text{all space} \psi^*(x) \hat{A} \psi(x) \, dx\right]^2,
\end{equation}
that is, the difference between the average value of $\hat{A}^2$ minus the square of the average value of $\hat{A}$.

\subsection{Uncertainty in the measurement of an eigenstate}
Suppose that a particle in a box is in an eigenstate of the Hamiltonian, $\psi_n(x)$, and ask: What is the average and uncertainty in the energy?
The average energy for state $\psi_n(x)$ is given by
\begin{equation}
\langle \hat{H} \rangle =  \int_\text{all space} \psi_n^*(x) \hat{H} \psi_n(x) \, dx,
\end{equation}
and we can now use the fact that $\psi_n(x)$ is an eigenstate of $\hat{H}$ with eigenvalue $E_n$,
\begin{equation}
\hat{H} \psi_n(x) = E_n \psi_n(x),
\end{equation}
to simplify this integral
\begin{equation}
\langle E \rangle = \langle \hat{H} \rangle =  \int_\text{all space} \psi_n^*(x) E_n \psi_n(x) \, dx 
=  E_n \underbrace{\int_\text{all space} \psi_n^*(x) \psi_n(x) \, dx}_{= 1} = E_n.
\end{equation}
In the last step we used the fact that the wave function is normalized to simplify the integral.
This results tells us that the average measurement will given an energy equal to $E_n$.
The standard deviation in the energy we measure is given by
\begin{equation}
\sigma^2_E
= \int_\text{all space} \psi_n^*(x) (\hat{H} - \langle \hat{H} \rangle)^2 \psi_n(x) \, dx.
\end{equation}
Note that the quantity inside the integral is zero
\begin{equation}
(\hat{H} - \langle \hat{H} \rangle)^2 \psi_n(x) = (\hat{H} - \langle \hat{H} \rangle) (\hat{H} - \langle \hat{H} \rangle)\psi_n(x) = (\hat{H} - \langle \hat{H} \rangle) \underbrace{\left[\hat{H} \psi_n(x) - E_n \psi_n(x)\right]}_{= 0},
\end{equation}
and hence, 
\begin{equation}
\sigma^2_E = 0
\end{equation}
This result tells us that there is no uncertainty when we measure the energy of an eigenstate, since every time we measure the energy we get as a result $E_n$.

\subsection{Superposition of states and measurements}
What happens when we measure the energy of a system that is not in an eigenstate of the Hamiltonian?

One of the postulates of quantum mechanics says that if we have some generic state $\phi(x)$ that is not an eigenstate of the Hamiltonian, we can always represent this state as a combination of eigenstates of $\hat{H}$. For example, in the case of the particle in a box we can write
\begin{equation}
\phi(x) = \sum_{n = 0}^\infty c_n \psi_n(x),
\end{equation}
where the numbers $c_n$ are complex and are often referred to as expansion coefficients or just coefficients.
These coefficients have norm less or equal to one
\begin{equation}
|c_n|^2 \leq 1 \quad \forall n,
\end{equation}
and their sum is equal to one
\begin{equation}
\sum_{n = 0}^\infty |c_n|^2 = 1.
\end{equation}
Quantum mechanics tells us that when we measure the energy of state $\phi(x)$ we will get as a result one of the eigenvalues of the Hamiltonian.
However, if we know the expansion coefficients of $\phi(x)$ we can say something more.
Indeed, the probability of measuring the energy $E_n$ is equal to the modulus square of the coefficient for eigenstate $\psi_n(x)$
\begin{equation}
\text{Probability of measuring } E_n = P_n = |c_n|^2.
\end{equation}
For example, consider the state
\begin{equation}
\phi(x) = \underbrace{\frac{1}{\sqrt{2}}}_{c_1} \psi_1(x) \underbrace{- i \frac{1}{\sqrt{2}}}_{c_3} \psi_3(x).
\end{equation}
Here we can immediately identify the coefficient for each state. We have $c_1 = \frac{1}{\sqrt{2}}$ and $c_3 = - i \frac{1}{\sqrt{2}}$, and all other coefficients are zero, $c_2 = c_4 = \ldots = 0$.
What is the probability of measuring $E_1$ and $E_3$? We can compute these quantities by taking the square of the wave function coefficients
\begin{equation}
\begin{split}
P_1 &= \left|\frac{1}{\sqrt{2}}\right|^2 = \frac{1}{2},\\
P_3 &= \left|-i\frac{1}{\sqrt{2}}\right|^2 = \frac{1}{2}.
\end{split}
\end{equation}

\end{document}