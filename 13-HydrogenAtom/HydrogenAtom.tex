% This work is licensed under the Creative Commons Attribution-NonCommercial 4.0 International License.
% To view a copy of this license, visit http://creativecommons.org/licenses/by-nc/4.0/
% or send a letter to Creative Commons, PO Box 1866, Mountain View, CA 94042, USA.

% !TEX TS-program = xelatex

\documentclass[../Main/chem331-notes.tex]{subfiles}
\begin{document}

\setcounter{section}{12}

\section{The hydrogen atom (incomplete)}
\subsection{The classical Coulomb potential}
In our treatment of the hydrogen atom we will start by assuming that the proton is fixed in space and we will only consider the wave function of the electron. The potential experienced by the electron orbiting around the proton is then
\begin{equation}
V(x,y,z) = - \frac{1}{4 \pi \epsilon_0} \frac{e^2}{r}
\end{equation}
where $\epsilon_0$ is the vacuum permittivity.
For convenience we will switch right away to atomic units, for which the term
\begin{equation}
\frac{e^2}{4 \pi \epsilon_0} = 1 \quad \text{(in atomic units)}.
\end{equation}
This leaves us with the much simpler form for the potential
\begin{equation}
V(x,y,z) = - \frac{1}{r}.
\end{equation}

\subsection{The Hamiltonian}
The Hamiltonian of the hydrogen atom in Cartesian coordinates is given by
\begin{equation}
\hat{H} = -\frac{1}{2} \nabla^2 - \frac{1}{r} = -\frac{1}{2} \left(\frac{\partial^2}{\partial x^2} + \frac{\partial^2}{\partial y^2} + \frac{\partial^2}{\partial z^2} \right) - \frac{1}{\sqrt{x^2 + y^2 + z^2}},
\end{equation}
and as we have seen before, it is more convenient to write it in spherical coordinates as
\begin{equation}
\hat{H} =  -\frac{1}{2} \left[ \frac{1}{r^2} \frac{\partial}{\partial r} \left( r^2 \frac{\partial }{\partial r} \right)
+ \frac{1}{r^2 \sin \theta} \frac{\partial}{\partial \theta} \left( \sin\theta \frac{\partial }{\partial \theta} \right)
+ \frac{1}{r^2 \sin^2 \theta} \frac{\partial^2}{\partial \phi^2} \right]  - \frac{1}{r}.
\end{equation}
Part of this Hamiltonian are related to angular momentum.
In spherical coordinates the angular momentum squared operator ($\hat{L}^2$) is given by the following equation (note that there is an imiplicit $\hbar^2$ factor that  goes away when we write $\hat{L}^2$ in atomic units)
\begin{equation}
\hat{L}^2 = - \left[\frac{1}{ \sin \theta} \frac{\partial}{\partial \theta} \left( \sin\theta \frac{\partial }{\partial \theta} \right)
+ \frac{1}{ \sin^2 \theta} \frac{\partial^2}{\partial \phi^2} \right].
\end{equation}
Using the definition of $\hat{L}^2$ in spherical coordinates it is possible to write the Hamiltonian in a more compact way
\begin{equation}
\hat{H} =  -\frac{1}{2} \left[ \frac{1}{r^2} \frac{\partial}{\partial r} \left( r^2 \frac{\partial }{\partial r} \right)
- \frac{\hat{L}^2}{r^2}  \right]  - \frac{1}{r}.
\end{equation}
To find the eigenvalues and eigenfunctions of $\hat{H}$ we attempt to write the solution in a factorized form
\begin{equation}
\psi(r,\theta,\phi) = R(r) Y_l^{m_l}(\theta,\phi),
\end{equation}
where $R(r)$ is the \textbf{radial wave function} and $Y_l^{m_l}(\theta,\phi)$ is a spherical harmonic.\mnote{To follow the conventional scheme to label the solution of the hydrogen atom here we use the symbol $m_l$ instead of $m$.}
If we plug this guess into the Schr\"{o}dinger equation we get
\begin{equation}
\begin{split}
\hat{H}\psi(r,\theta,\phi) & =  -\frac{1}{2} \left[ \frac{1}{r^2} \frac{\partial}{\partial r} \left( r^2 \frac{\partial }{\partial r} \right)
- \frac{\hat{L}^2}{r^2}  \right]R(r) Y_l^{m_l}(\theta,\phi)  - \frac{R(r) Y_l^{m_l}(\theta,\phi)}{r} \\
& = -\frac{1}{2} Y_l^{m_l}(\theta,\phi) \left[ \frac{1}{r^2} \frac{\partial}{\partial r} \left( r^2 \frac{\partial  }{\partial r} \right)
- \frac{l(l+1)}{r^2}  \right] R(r)  - \frac{R(r) Y_l^{m_l}(\theta,\phi)}{r} \\
& = E R(r) Y_l^{m_l}(\theta,\phi).
\end{split}
\end{equation}
Note that all the operator components that act on $\theta$ and $\phi$ are contained in the $\hat{L}^2$ operator. Since $Y_l^{m_l}(\theta,\phi)$ is an eigenfunction of $\hat{L}^2$, in the second step we were able to replace $\hat{L}^2Y_l^{m_l}(\theta,\phi)$ with $l(l+1)Y_l^{m_l}(\theta,\phi)$.
Since $Y_l^{m_l}(\theta,\phi)$ multiplies both sides of this equation, we can remove this factor and obtain an equation for the radial wave function alone
\begin{equation}
\Big[\underbrace{
 -\frac{1}{2r^2} \frac{\partial}{\partial r} \left( r^2 \frac{\partial  }{\partial r} \right)
 }_{\text{kinetic}}
+ \underbrace{
\frac{l(l+1)}{2r^2} - \frac{1}{r}
}_{\text{potential}}
\Big] R(r) = E R(r).
\end{equation}
We can identify two contributions to this equation. The first one contains derivatives with respect to $r$ and can be associated with the kinetic energy.
The second term
 \begin{equation}
V_l(r) = \frac{l(l+1)}{2r^2} - \frac{1}{r},
\end{equation}
can be considered an effective Coulomb potential. It contains the correct $- r^{-1}$ term that we expect from the classical Coulomb potential plus a term proportional to $r^{-2}$.
The latter, is due to angular momentum and counters the effect of Coulomb attraction.
Since the effective potential depends on $l$, we have to solve a different radial Schr\"{o}dinger equation for each value of $l$.

Note that the radial equation is sometimes expressed as a function of the variable
\begin{equation}
\rho = 2r,
\end{equation}
so that the left hand side can be written as\mnote{
$\frac{\partial}{\partial r} = \frac{\partial\rho}{\partial r} \frac{\partial}{\partial \rho} = \frac{2}{n} \frac{\partial}{\partial \rho}$.
}
\begin{equation}
-\frac{1}{2r^2} \frac{\partial}{\partial r} \left( r^2 \frac{\partial  }{\partial r} \right) + \frac{l(l+1)}{2r^2} - \frac{1}{r}
= -2 \frac{\partial^2}{\partial \rho^2}
-\frac{4}{\rho} \frac{\partial}{\partial \rho} + 2\frac{l(l+1)}{\rho^2} - \frac{2}{\rho},
\end{equation}
and can be used to write the Schr\"{o}dinger equation as
\begin{equation}
\left[
-\frac{\partial^2 }{\partial \rho^2}
-\frac{2}{\rho}\frac{\partial }{\partial \rho}
-\frac{1}{\rho} + \frac{l(l+1)}{\rho^2} \right] R(\rho) = E' R(\rho),
\end{equation}
where the energy $E' = E / 2$.

The solution of the radial equation is quite involved and we will jump directly to the solution. The eigenvalues ($E_n$ of the radial equation are given by
\begin{iequation}
E_n = - \frac{1}{2 n^2}  \, \mathrm{a.u.},
\end{iequation}
with the energy being expressed in atomic units (1 Hartree = 627.51 kcal mol$^{-1}$).
The corresponding eigenfunctions are
\begin{equation}
R_{n,l}(\rho) = \sqrt{\left(\frac{2}{n} \right)^{3} \frac{ (n-l-1)! }{ 2n (n+l)!} }\left(\frac{\rho}{n}\right)^l e^{-\rho/2n} L_{n+l}^{2l+1}(\rho/n),
\end{equation}
%\begin{equation}
%R_{n,l}(r) = N_{n,l} \left(\frac{\rho}{n}\right)^l L_{n,l}(\rho) e^{-\rho/2n},
%\end{equation}
where $n$ is called the \textbf{principal quantum number} and can take the following values
\begin{equation}
n = 1, 2, 3, \ldots
\end{equation}
The quantity $L_{n+l}^{2l+1}(x)$ indicates \textbf{associated Laguerre polynomials}.
Converting from $\rho$ to $r$ we can write the radial equation in an equivalent way as
\begin{equation}
R_{n,l}(r) = \sqrt{\frac{ (n-l-1)! }{ 2n (n+l)!} } \left(\frac{2}{n} \right)^{l + 3/2} r^l e^{-r/n} L_{n+l}^{2l+1}(2r/n).
\end{equation}
The first few associated Laguerre polynomials are
\begin{equation}
\begin{split}
L_0^\alpha(x) & = 1 \\
L_1^\alpha(x) & = \alpha - x + 1 \\
L_2^\alpha(x) & = \frac{x^2}{2} - (\alpha + 2) x + \frac{(\alpha + 2)(\alpha + 1)}{2}.
\end{split}
\end{equation}

The first few radial wave functions are
\begin{equation}
\begin{split}
R_{1,0}(r) & = 2 e^{-r}, \\
R_{2,0}(r) & =\frac{2 - r}{2 \sqrt{2}} e^{-r/2}, \\
R_{2,1}(r) & =\frac{r}{2 \sqrt{6}} e^{-r/2}, \\
R_{3,0}(r) & =\frac{27 - 18 r + 2 r^2}{81 \sqrt{3}} e^{-r/3}.
\end{split}
\end{equation}

\subsection{Quantum numbers and degeneracy}
Combining the results of the previous section we can write the wave function of the hydrogen atom in terms of three quantum numbers
\begin{equation}
\psi_{nlm_l}(r,\theta,\phi) = R_{nl}(r) Y_l^{m_l}(\theta,\phi),
\end{equation}
The principal quantum number $n$ is a positive integer and is connected to the total energy of a quantum state. 
In addition to the principal quantum number $n$ the wave function depends on the \textbf{angular momentum quantum number} $l$.  The allowed values of this quantum number are
\begin{iequation}
l = 0, 1, \ldots, n-1,
\end{iequation}
and it determines the total angular momentum squared of the hydrogen atom since
\begin{equation}
\langle \hat{L}^2 \rangle = \hbar^2 l(l + 1).
\end{equation}
In the convention used to label states of the hydrogen atom we assign symbols to different values of $l$. For $l = 0$ we label the state as $s$, $l = 1$ corresponds to $p$, $l= 2$ to $d$, and $l = 3$ to $f$. After $l=3$ we use letters in alphabetical order ($g$, $h$, $i$,\ldots).
The \textbf{magnetic quantum number} $m_l$ can take any of the following values
\begin{iequation}
m = 0, \pm 1, \pm 2,  \ldots, l.
\end{iequation}
For each value of $l$ there are $2l +1$ possible values of $m_l$. 
The magnetic quantum number is related to the projection of angular momentum on the $z$ axis
\begin{equation}
\langle \hat{L}_z \rangle = \hbar m_l.
\end{equation}

The states of the hydrogen atom are usually written as $nl$, using the corresponding symbol for $l$.
For example, the ground state ($n=0$, $l=0$) is designated $1s$. The state ($n=1$, $l=0$) is indicates with $2s$. The state ($n=2$, $l=1$) corresponds to $2p$ and ($n=3$, $l=2$) to $3d$.

Since the energy depends only on the quantum number $n$, states with different values of $l$ and $m_l$ but same $n$ will be degenerate.
For each value of $l$ there are $2l +1$ possible values of $m_l$. If we sum up all the possible states for a fixed $l$ we have that the total number of states for a given $n$ is equal to\mnote{Here we use the equation for the sum of the first $n$ integers, $\sum_{k = 1}^{n} k = 1+2+\ldots + n = n (n+1) / 2 $.}
\begin{equation}
\text{Degeneracy of level } n = \sum_{l = 0}^{n-1} (2l +1) = 2 \sum_{l = 0}^{n-1} l + \sum_{l = 0}^{n-1} 1
= 2 \frac{(n - 1) n}{2} + n = n^2.
\end{equation}

\subsection{Eigenfunctions of the hydrogen atom}
See next update to these notes.

\end{document}





